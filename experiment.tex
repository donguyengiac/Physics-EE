From the final expression relating $\Delta f$ and $l_n$, knowing all other constants I can plot the graph for this. $\Delta f$ is the dependent variable, and $l_n$ is the independent variable. I expect the intonation difference $\Delta f$ to be higher when the fretting distance is closer to the bridge ($l_n$ closer to 0), and lower closer to the nut, in the first few frets (larger values of $l_n$). The general shape of the graph can be compared with my expectation as a sanity check. It is impossible to linearize the equation, therefore I will have to resort to collecting the data, plotting the graph, then confirming the relationship using a regression method to see how well the data fits the model. \par
The choice of guitar I will use for this experiment is a Fender Stratocaster. This is arguably the most iconic and popular guitar in history, known for its versatility and playability. Therefore I pick this guitar because there is a wealth of information available on it, to make it easier to find data to support my research or to compare my results to other studies. Also this will make it easier to control variables in my experiment and ensure consistent results. Another benefit is to ensure ease of replicability of the experiment. \par
The choice of string I will use is a G string from D'Addario's set of Nickel Wound Regular Light Gauge - EXL110-10P that I have at home. I choose the G string because it is the thickest string in the set that is a plain unwound string. This is because from my experience the thicker strings will make the effects of the intonation shift more noticeable. 