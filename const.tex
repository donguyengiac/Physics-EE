From the equation (\ref{eqn39}), the constants I need to determine are $f_0$, $\rho$, $Y$, $a$, $b$ and $l$. 
\begin{itemize}
    \item $l$ is the scale length of the guitar. For a Fender Stratocaster, it is 25.5 inches (64.77 cm). \cite{scale} I round this up to 64.8 cm (3s.f). From this information, I can adjust the bridge saddle position so that the scale length matches 64.8 cm (0.648 m)
    \item Initial frequency $f_0$ can be measured directly in the experiment. The frequency we aim for is the frequency of G string on the guitar for a standard A440 tuning system, $G_3$ at 196 Hz \cite{freq_chart}
    \item Young's modulus $Y$ is dependent on the specifications of the string material. Nearly all electric guitar strings follow the ASTM-A228 manufacturing standards for steel music wire, and the value of $Y$ is \SI{210}{\giga\pascal} (\SI{2.10e11}{\pascal}). \cite{astm} 
    \item $\rho$, the density of the string material, is determined according to the ASTM-A228 standards: $\rho = \SI{7.80e3}{\kg.m^{-3}}$. \cite{astm}
    \item $a$ and $b$: it is very hard to measure these directly, as they are distance from the nut and saddle to the top of the fret, not the height itself. Therefore, I can only set up the guitar indirectly according to recommended values and calculate them afterwards. There are a lot of resources online on how to set up the guitar. I choose to follow the instructions by Stewmac \cite{stewmac}, a reputable online guitar retailer, and take the average value of the action (distance between string and top of fret) for the \nth{1} fret to be 0.016" (0.406 mm) and \nth{12} fret to be 0.070" (1.78 mm). I change the adjustable saddle height and file down/shim up the nut height to match these values. From there I can calculate the values of $a$ and $b$ as illustrated by Figure \ref{fig3}: \par
\end{itemize}
\begin{figure}[!htbp]
    \includegraphics[width = \textwidth]{./ee/fig4.png}
    \caption{Diagram to calculate values of $a$ and $b$} \label{fig3}
\end{figure}
Since fret \nth{12} is exactly in the middle of $l$ ($l_{12} = \frac{l}{2} = \frac{.648}{2} = .324$ m) we get:
\begin{align*}
    a &= h_{12} + x \\
    b &= h_{12} - x
\end{align*}
From the luthier formula (\ref{eqn2}):
\begin{align*}
    l_1 &= \frac{l}{2^{\frac{1}{12}}} \\
        &= \frac{0.648}{2^{\frac{1}{12}}} \\
        &= \SI{0.612}{\meter}
\end{align*}
and we also get the ratio in Figure \ref{fig3}
\begin{align*}
    \frac{x}{y} &= \frac{l-l_{12}}{l_1-l_{12}} \\
    x &= y\frac{l-l_{12}}{l_1-l_{12}} \\
    &= (h_{12}-h_1)\frac{l-l_{12}}{l_1-l_{12}} \\
    &= (1.78-0.406) \cdot 10^{-3} \cdot \frac{0.648-0.324}{0.612-0.324} \\
    &= \SI{1.55e-3}{\meter}
\end{align*}
Therefore
\begin{align*}
    a &= 1.78 + 1.55 = \SI{3.33e-3}{\meter} \\
    b &= 1.78 - 1.55 = \SI{1.30e-4}{\meter}
\end{align*}    

