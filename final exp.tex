Finally, we can substitute (\ref{eqn29}) into (\ref{eqn20}) to get the expression between the intonation shift $\Delta f$ and the fret position $l_n$
\begin{align}
    \Delta f &= \frac{1}{4l_n} \sqrt{\frac{T}{\mu}} \left( \frac{AY}{2Tl} \left( \frac{b^2}{l-l_n} + \frac{a^2}{l_n} \right) - \frac{a^2}{{l_n}^2} \right) \label{eqn30}
\end{align}
From (\ref{eqn4}) we get
\begin{equation*}
    \sqrt{\frac{T}{\mu}} = 2lf_0
\end{equation*}
and 
\begin{equation*}
    T = 4\mu l^2{f_0}^2    
\end{equation*}
Subbing into (\ref{eqn30})
\begin{equation}
    \Delta f = \frac{l f_0}{2l_n} \left( \frac{AY}{8\mu l^3 {f_0}^2} \left( \frac{b^2}{l-l_n} + \frac{a^2}{l_n} \right) - \frac{a^2}{{l_n}^2} \right) \label{eqn31}
\end{equation}
$\mu$, the linear density of the string, is simply the mass per unit length of the string:
\begin{equation*}
    \mu = \frac{m}{l} = \frac{\rho V}{l} = \frac{\rho l A}{l} = \rho A
\end{equation*}
where $m$ is the mass of the vibrating string section, $V$ is its volume, and $\rho$ is the volumetric density of the material. Subbing this into (\ref{eqn31}):
\begin{align}
    \Delta f &= \frac{l f_0}{2l_n} \left( \frac{AY}{8\rho A l^3 {f_0}^2} \left( \frac{b^2}{l-l_n} + \frac{a^2}{l_n} \right) - \frac{a^2}{{l_n}^2} \right) \label{eqn32} \\
    &= \frac{l f_0}{2l_n} \left( \frac{Y}{8\rho l^3 {f_0}^2} \left( \frac{b^2}{l-l_n} + \frac{a^2}{l_n} \right) - \frac{a^2}{{l_n}^2} \right) \label{eqn33}
\end{align}