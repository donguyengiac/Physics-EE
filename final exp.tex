Finally, we can substitute (\ref{eqn37}) into (\ref{eqn32}) to get the expression between the intonation shift $\Delta f$ and the fret position $l_n$
\begin{align}
    \Delta f &= \frac{1}{4l_n} \sqrt{\frac{T}{\mu}} \bigg( \frac{AY}{2Tl \big(1 + \frac{3}{2}(\frac{a-b}{l})^2\big)} \left( \frac{b^2}{l-l_n} + \frac{a^2}{l_n} - \frac{(a-b)^2}{l} \right) + \Big(\frac{a-b}{l}\Big)^2 - \Big(\frac{a}{l_n}\Big)^2 \bigg) \label{eqn38}
\end{align}
From (\ref{eqn3}) we get
\begin{align*}
    \sqrt{\frac{T}{\mu}} &= 2lf_0 \\
    \implies T &= 4\mu l^2{f_0}^2    
\end{align*}
$\mu$ is simply the mass per unit length of the string:
\begin{equation*}
    \mu = \frac{m}{l_s} = \frac{\rho V}{l_s} = \frac{\rho l_s A}{l_s} = \rho A
\end{equation*}
where $m$ is the mass of the vibrating string section, $V$ is its volume, and $\rho$ is the volumetric density of the material. Therefore:
\begin{equation*}
    T = 4\rho A l^2{f_0}^2    
\end{equation*}
Substituting all this into (\ref{eqn38}) we get the final expression between the intonation shift $\Delta f$ and the fret position $l_n$ relating all our known variables:
\begin{align}
    \Delta f &= \frac{l f_0}{2l_n} \left( \frac{Y}{8\rho l^3 {f_0}^2 (1 + \frac{3}{2}(\frac{a-b}{l})^2)} \left( \frac{b^2}{l-l_n} + \frac{a^2}{l_n} - \frac{(a-b)^2}{l} \right) + \frac{(a-b)^2}{l^2} - \frac{a^2}{{l_n}^2} \right) \label{eqn39}
\end{align}