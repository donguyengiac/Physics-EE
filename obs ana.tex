From Figure \ref{fig9} we see the data points fit the function quite well and follow the general shape of the curve. The horizontal error bars are too small to see. The curve passes through all vertical error bars except for the data point at fret 15, which is still close to the curve but error bars are too small. Noticeably, although data points of lower frets stay pretty close to the curve, for higher frets (around frets 14-16) $\Delta f$ are relatively higher than the predicted values. One possible cause is due to the capo placement. Because of the larger neck shape around this position where it connects with the guitar body, it is rather difficult to place the capo squarely on the fret and pressing directly down on the string. This might cause it to pull the string a bit sideways or push it down behind the fret, increasing the tension and making the frequency go up. \par
The goodness-of-fit test I will perform on the data is the Standard Error of the Estimate (SEOE). 
\begin{equation*}
    \text{SEOE} = \sqrt{\frac{\sum{(Y-Y')^2}}{n}}
\end{equation*}
Where $Y$ is the actual value of the data, $Y'$ is the value from the function, and $n$ is the number of data points. \cite{lane} \par
The correlation value of SEOE is the standard deviation of the residual values. The closer this standard deviation is to 0, the lower the errors and the better the function fits the data (higher correlation). I choose this test because it returns a normalized result and provides a clear measure on the quality of the correlation. Performing the test on the processed data points of $\Delta f$ and $l_n$, I get a value of SEOE $= 0.105$. This is quite close to 0, indicating a low deviation of errors, so I believe the data fits the model well.