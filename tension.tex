When the guitar string is under tension $T_a$, it causes the string to stretch an amount $\Delta l_0$ as compared to the unstretched length, which is the same as the effective vibrating strength of the string $l_s$. This relationship can be determined as:
\begin{equation}
    T_a = \frac{AY\Delta l_0}{l_s}\label{eqn18}
\end{equation} 
Where $A$ is the cross-sectional area of the string, and $Y$ is the Young's modulus of the material. \cite{polak}

When we press down on it, the whole string stretches accordingly by a small amount $\Delta l$. Assuming there is no friction on the point of contact, the tension in the whole string increases by an amount $\Delta T$:
\begin{equation}
    T_a + \Delta T = \frac{AY(\Delta l_0 + \Delta l)}{l_s}\label{eqn19}
\end{equation}
Thus the relationship between $\Delta T$ and $\Delta l$ can be determined:
\begin{equation}
    \Delta T = \frac{AY\Delta l}{l_s}\label{eqn20}
\end{equation} 
It is very hard to observe and measure $\Delta l$ directly, so we can try rewriting it in terms of other variables that can be determined. From Figure \ref{fig2},  
\begin{align}
    \Delta l &= (l_b + l_n') - l_s \\
    &= \sqrt{(l-l_n)^2+b^2} + \sqrt{l_n^2+a^2} - l_s \\
    &= (l-l_n)\left(1+\left(\frac{b}{l-l_n}\right)^2\right)^{\frac{1}{2}} + l_n\left(1+\left(\frac{a}{l_n}\right)^2\right)^{\frac{1}{2}} - l_s
\end{align}
Once again, we can approximate this to the first order since $a << l_n$ and $b << l-l_n $:
\begin{align}
    \Delta l &\approx (l-l_n)\left(1+\frac{b^2}{2(l-l_n)^2}\right)+ l_n\left(1+\frac{a^2}{2l_n^2}\right) - l_s \label{eqn24} \\
    &= (l - l_n) + \frac{b^2}{2(l-l_n)} + l_n + \frac{a^2}{2l_n} - l_s \label{eqn25} \\
    &= \frac{b^2}{2(l-l_n)} + \frac{a^2}{2l_n} - (l_s - l) \label {eqn26}
\end{align}
From Figure \ref{fig2}, we can expand and approximate $l_s$ up to first order ($a-b << l$):

\begin{align}
    l_s &= \sqrt{l^2 + (a-b)^2}\\
    &= l\bigg(1 + \bigg(\frac{a-b}{l}\bigg)^2\bigg)^\frac{1}{2} \label{eqn28}\\
    &\approx l\bigg(1 + \frac{1}{2}\bigg(\frac{a-b}{l}\bigg)^2\bigg)\\
    &= l + \frac{(a-b)^2}{2l}
\end{align}
Therefore from (\ref{eqn26}):
\begin{align}
    \Delta l &= \frac{b^2}{2(l-l_n)} + \frac{a^2}{2l_n} - \frac{(a-b)^2}{2l}
\end{align}
Substituting this into (\ref{eqn20}) we get
\begin{equation}
    \Delta T = \frac{AY}{2l_s} \left( \frac{b^2}{l-l_n} + \frac{a^2}{l_n} - \frac{(a-b)^2}{l} \right) \label{eqn32}
\end{equation}
Which can be substituted into (\ref{eqn17}):
\begin{align}
    \Delta f &= \frac{1}{4l_n} \sqrt{\frac{T}{\mu}} \bigg( \frac{AY}{2l_s T_a} \left( \frac{b^2}{l-l_n} + \frac{a^2}{l_n} - \frac{(a-b)^2}{l} \right) + \bigg(\frac{a-b}{l}\bigg)^2 - \bigg(\frac{a}{l_n} \bigg)^2 \bigg) \label{eqn33}
\end{align}
We need to write $l_s T_a$ in terms of $l$, $T$ and other variables. From (\ref{eqn11}) and (\ref{eqn28}), once again using a first order approximation since $a-b << l$:

\begin{align}
    T_a &= T\bigg(\frac{l_s}{l}\bigg)^2 \\
    \implies l_s T_a &= T \frac{l_s^3}{l^2} \\
    &= Tl \bigg(1 + \bigg(\frac{a-b}{l}\bigg)^2\bigg)^\frac{3}{2} \\
    &\approx Tl \bigg(1 + \frac{3}{2}\bigg(\frac{a-b}{l}\bigg)^2\bigg) \label{eqn37}
\end{align}