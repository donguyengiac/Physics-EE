In the guitar community, when comparing the build quality of different guitar brands, one of the key determining points is whether a guitar can keep good intonation or not. Intonation refers to the accuracy of the instrument’s pitch across all frets. Slight deviations in intonation can have a significant impact on the overall sound quality, especially when playing chords, as multiple slightly-out-of-tune notes together can clash and produce muffled, discordant sounding chords, most notably when playing high up the neck (above $11-12^{th}$ fret). This intonation problem has been a topic of debate in many guitar forums online, with a lot of hypotheses on why it happens and discussions on how to set a perfect intonation. There has been limited formal studies of the subject, the two most notable ones came from the works of classical guitar luthiers (\cite{bartolinis} and \cite{byers}), the former of which is considered seminal, yet seems to have been lost on the internet. In terms of physics investigation there have been general basic studies on intonation yet very little to no detailed analysis \cite{varieschi}. The model used in both \cite{byers} and \cite{varieschi} are derived from the original model by \cite{bartolinis}, but I think they don't reflect the real world behavior really well. Therefore, in this investigation I want to explore how intonation works, what causes intonation problems, and from there make observations on some solutions to achieve good intonation. This investigation is not only significant for my hobby of playing the guitar, but it is also a topic of interest for many musicians and music lovers worldwide. By delving deeper into the science behind the intonation, I hope to not only improve my skills as a guitarist but also contribute to the broader community of music enthusiasts.