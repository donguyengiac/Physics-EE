In the guitar community, when comparing the build quality of different electric guitars, one of the key determining points is a guitar's ability to keep good intonation. Intonation refers to the accuracy of the instrument’s pitch across all frets. Slight deviations in intonation can have a significant impact on the overall sound quality, especially when playing chords, as multiple slightly-out-of-tune notes together can clash and produce muffled, discordant sounding chords, most notably when playing high up the neck (above $\nth{11}-\nth{12}$ fret). This problem has been a topic of debate in many guitar forums online, with a lot of hypotheses on why it happens and discussions on how to set a perfect intonation. There has been extremely limited formal studies of the subject, the two most notable ones came from the works of classical guitar luthiers \footnote{Craftsmen who build or repair stringed instruments} (\cite{bartolinis} and \cite{byers}), the former of which is considered seminal, yet seems to have been lost on the internet. In terms of physics investigation there have been general basic studies on intonation yet very little detailed analysis \cite{varieschi}. The model used in both \cite{byers} and \cite{varieschi} are derived from the original model by \cite{bartolinis}, but I think they do not reflect the real world behavior really well. Therefore, in this investigation I want to develop a different model to explore how intonation works, what causes intonation problems, and propose some solutions to achieve good intonation.