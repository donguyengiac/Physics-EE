In the guitar community, when comparing the build quality of different guitar brands, one of the key determining points is whether a guitar has good intonation or not. Intonation refers to the accuracy of the instrument’s pitch across all frets.  Slight deviations in intonation can have a significant impact on the overall sound quality. This change in intonation often starts around fret 7-8 and above, and most noticeable especially when I play chords high up the neck with a distortion pedal. This is because the higher harmonic frequencies from the distortion will clash with the other notes in the chord when the intonation is not perfect, resulting in a very muffled, discordant sounding chord. This intonation problem has been a topic of debate in many guitar forums online, with a lot of hypotheses on why it happens and discussions on how to set a perfect intonation. Therefore this motivated me to combine it with my love of physics and investigate the principles behind this problem. I want to explore how intonation works, why the note intonation is “off” when you fret a string on high frets, and from there propose solutions to achieve perfect intonation. This investigation is not only significant for my hobby of playing the guitar, but it is also a topic of interest for many musicians and music lovers worldwide. By delving deeper into the science behind the intonation, I hope to not only improve my skills as a guitarist but also contribute to the broader community of music enthusiasts.